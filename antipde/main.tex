\documentclass[UTF8]{ctexart}
\usepackage{graphicx}
\usepackage{amsmath}
\usepackage{float}
\usepackage{booktabs}
\usepackage{geometry}
\usepackage{caption}
\usepackage{subcaption}
\usepackage[hidelinks]{hyperref}

\geometry{a4paper, scale=0.8}

\title{\textbf{课后练习:\\DeepONet 反导数算子学习:架构与采样策略影响分析报告}}
\author{刘昭阳 25215133 \\ 中山大学数学学院(珠海)}
\date{\today}

\begin{document}

\maketitle

\begin{abstract}
本报告旨在深入探究 DeepONet 在学习反导数算子任务中的性能表现。我们不仅评估了基准模型的精度,还系统地分析了网络深度(Network Depth)和域内采样密度(Sampling Density)对模型泛化能力的影响。实验结果表明,基准模型在全量数据下表现最佳,测试集相对误差仅为 $1.47\%$。盲目增加网络深度会导致过拟合,而稀疏的域内采样则会显著削弱模型捕捉函数细节的能力。
\end{abstract}
\newpage
\tableofcontents
\newpage
\section{引言}
\subsection{背景}
在科学计算和工程领域,偏微分方程(PDE)和积分方程的求解至关重要。传统的数值方法(如有限元法 FEM、有限差分法 FDM)虽然精度高,但计算成本高昂,且通常针对特定方程和边界条件进行求解。近年来,基于深度学习的算子学习(Operator Learning)方法应运而生,旨在学习函数空间之间的映射,从而实现对一类方程的快速求解。

\subsection{DeepONet 简介}
DeepONet(Deep Operator Network)是由 Lu 等人提出的一种通用的算子学习框架,其理论基础是算子逼近定理。DeepONet 通过两个子网络——Branch Net(分支网络)和 Trunk Net(主干网络)——分别编码输入函数和查询位置,从而实现对非线性算子的有效逼近。

\subsection{任务定义}
本实验聚焦于一个经典的算子学习任务:\textbf{反导数算子(Antiderivative Operator)}。给定一个定义在 $[0, 1]$ 区间上的函数 $u(x)$,我们的目标是学习算子 $G$,使得 $G(u) = v$,其中 $v(x)$ 是 $u(x)$ 的反导数,即:
\begin{equation}
    v(x) = \int_0^x u(\tau) d\tau
\end{equation}
这是一个典型的积分算子学习问题。

\section{方法论}

\subsection{DeepONet 架构}
DeepONet 的核心思想是将算子 $G$ 近似为:
\begin{equation}
    G(u)(y) \approx \sum_{k=1}^p b_k(u(x_1), u(x_2), \dots, u(x_m)) \cdot t_k(y) + b_0
\end{equation}
其中:
\begin{itemize}
    \item $u(x_1), \dots, u(x_m)$ 是输入函数 $u$ 在 $m$ 个固定传感器位置上的离散值。
    \item $b_k(\cdot)$ 是 Branch Net 的输出,负责提取输入函数的特征。
    \item $t_k(\cdot)$ 是 Trunk Net 的输出,负责提取查询位置 $y$ 的特征。
    \item $p$ 是两个子网络输出层的神经元个数。
\end{itemize}

\subsection{损失函数}
我们采用均方误差(MSE)作为损失函数来训练网络参数 $\theta$:
\begin{equation}
    \mathcal{L}(\theta) = \frac{1}{N} \sum_{i=1}^N \sum_{j=1}^P |G_\theta(u^{(i)})(y_j^{(i)}) - v^{(i)}(y_j^{(i)})|^2
\end{equation}
其中 $N$ 是训练样本数,$P$ 是每个样本的域内采样点数。

\section{实验设置}

\subsection{数据集}
数据集包含成对的函数 $(u, v)$,其中 $u$ 是输入函数,$v$ 是对应的反导数。
\begin{itemize}
    \item \textbf{训练集}:加载自 \texttt{antiderivative\_aligned\_train.npz}。
    \item \textbf{测试集}:加载自 \texttt{antiderivative\_aligned\_test.npz}。
\end{itemize}

\subsection{模型架构变体}
为了全面评估 DeepONet 的鲁棒性,我们构建了三种不同的实验配置,如表 \ref{tab:configs} 所示。

\begin{table}[H]
    \centering
    \caption{实验配置详情}
    \label{tab:configs}
    \begin{tabular}{lcccc}
        \toprule
        \textbf{实验组} & \textbf{Branch Net 结构} & \textbf{Trunk Net 结构} & \textbf{域内采样点数 ($N$)} & \textbf{说明} \\
        \midrule
        \textbf{Baseline (基准)} & [100, 40, 40] & [1, 40, 40] & 100 & 标准配置 \\
        \textbf{Deeper Net (深层)} & [100, 40, 40, 40] & [1, 40, 40, 40] & 100 & 增加网络深度 \\
        \textbf{Sparse (稀疏)} & [100, 40, 40] & [1, 40, 40] & 20 & 仅使用 20\% 采样点 \\
        \bottomrule
    \end{tabular}
\end{table}

所有模型均使用 ReLU 激活函数和 Glorot normal 初始化,采用 Adam 优化器(学习率 0.001)训练 10,000 次迭代。

\section{结果与分析}

\subsection{训练动态分析}
图 \ref{fig:loss_comparison} 展示了三个模型在训练过程中的 Loss 下降曲线(对数坐标)。

\begin{figure}[H]
    \centering
    \includegraphics[width=0.8\textwidth]{figures/loss_comparison.png}
    \caption{不同模型的训练损失历史对比 (Log Scale)}
    \label{fig:loss_comparison}
\end{figure}

从图中可以观察到:
\begin{itemize}
    \item \textbf{Baseline (蓝色)}:收敛最为平稳,且最终 Loss 达到了较低的水平,表明优化过程顺利。
    \item \textbf{Deeper Net (橙色)}:虽然最终 Loss 也很低,但在训练初期和中期出现了较大的震荡。这可能是因为更深的网络带来了更复杂的损失地形(Loss Landscape),使得优化器更难找到全局最优解。
    \item \textbf{Sparse (绿色)}:由于数据量减少,Loss 曲线整体高于其他两组,且收敛速度较慢,表明模型难以从稀疏数据中学习到有效的算子映射。
\end{itemize}

\subsection{定量性能评估}
表 \ref{tab:results} 列出了各模型在测试集上的最终性能指标。

\begin{table}[H]
    \centering
    \caption{最终性能指标对比}
    \label{tab:results}
    \begin{tabular}{lccc}
        \toprule
        \textbf{模型} & \textbf{Train Loss} & \textbf{Test Loss} & \textbf{Test Metric (L2 Rel Error)} \\
        \midrule
        Baseline & $7.61 \times 10^{-6}$ & $1.54 \times 10^{-5}$ & $\mathbf{1.47\%}$ \\
        Deeper Net & $3.38 \times 10^{-6}$ & $2.61 \times 10^{-5}$ & $2.11\%$ \\
        Sparse Sampling & $3.41 \times 10^{-6}$ & $1.75 \times 10^{-5}$ & $1.70\%$ \\
        \bottomrule
    \end{tabular}
\end{table}

\textbf{分析}:
\begin{itemize}
    \item \textbf{Baseline 表现最佳}:在测试集上取得了最低的相对误差 ($1.47\%$),且 Train Loss 与 Test Loss 差距最小,泛化能力最强。
    \item \textbf{Deeper Net 过拟合}:虽然 Train Loss 较低($3.38 \times 10^{-6}$),但 Test Loss 显著高于 Baseline,Test Metric 也最差 ($2.11\%$)。这表明在当前数据规模下,增加网络深度导致了严重的过拟合,模型记住了训练数据但未能很好地泛化。
    \item \textbf{Sparse Sampling 欠拟合}:虽然 Train Loss 极低(可能是因为数据点少容易拟合),但 Test Metric 较差 ($1.70\%$)。由于采样点不足,模型无法捕捉函数的高频细节,导致在未见过的测试点上预测误差较大。
\end{itemize}

\subsection{预测结果可视化}
为了直观展示模型性能,我们选取了三个不同的测试样本,对比了各模型的预测曲线(图 \ref{fig:pred_multi})。

\begin{figure}[H]
    \centering
    \includegraphics[width=1.0\textwidth]{figures/prediction_comparison_multi.png}
    \caption{三个测试样本的预测结果对比}
    \label{fig:pred_multi}
\end{figure}

可以看出,Baseline 模型(红色虚线)与真实值(黑色实线)贴合最为紧密。Deeper Net(绿色虚线)在某些局部区域(如波峰波谷)出现了明显的偏差,这与其过拟合的特性一致。

\subsection{误差分布分析}
图 \ref{fig:error_dist} 展示了测试集上所有样本的相对 L2 误差分布直方图。

\begin{figure}[H]
    \centering
    \includegraphics[width=0.8\textwidth]{figures/error_distribution.png}
    \caption{测试集相对 L2 误差分布直方图}
    \label{fig:error_dist}
\end{figure}

Baseline 模型的误差分布最集中于 0 附近,且长尾效应最小。Deeper Net 的分布明显右移,说明存在更多的高误差样本。

\subsection{绝对误差沿定义域分布}
为了更细致地观察模型在定义域内的表现,我们绘制了第一个测试样本在 $[0, 1]$ 区间上的绝对误差分布(图 \ref{fig:abs_error})。

\begin{figure}[H]
    \centering
    \includegraphics[width=0.8\textwidth]{figures/absolute_error_sample.png}
    \caption{单样本绝对误差沿定义域分布 (Log Scale)}
    \label{fig:abs_error}
\end{figure}

从图中可以看出,Baseline 模型的误差(蓝色线)在整个定义域内都保持在较低水平。而 Deeper Net(橙色线)在某些位置的误差会出现尖峰,这进一步证实了其预测的不稳定性。

\section{讨论}
\subsection{网络深度的权衡}
本实验结果表明,对于相对简单的反导数算子,盲目增加网络深度并不能带来性能提升。相反,过深的网络增加了参数量,使得模型更容易过拟合,同时也增加了训练难度。在实际应用中,应根据任务的复杂度和数据量来选择合适的网络深度,或者引入正则化技术(如 Dropout、L2 正则化)来缓解过拟合。

\subsection{采样策略的重要性}
稀疏采样实验揭示了数据质量对算子学习的关键影响。根据奈奎斯特采样定理,如果采样频率低于信号最高频率的 2 倍,就会发生混叠,导致无法完美重构信号。在算子学习中,如果域内采样点过少,模型就无法感知输入函数的细微变化,从而限制了其逼近精度。因此,保证足够的采样密度是提高模型性能的必要条件。

\section{结论}
本实验通过对比分析得出以下结论:
\begin{enumerate}
    \item \textbf{架构选择}:对于反导数算子学习任务,[40, 40] 的隐藏层规模已经足够。盲目加深网络不仅增加了计算开销,还导致了泛化性能下降。
    \item \textbf{采样重要性}:域内采样密度对 DeepONet 的性能有直接影响。稀疏采样会丢失函数信息,限制模型的精度上限。
    \item \textbf{最佳实践}:在实际应用中,应优先保证数据的质量和采样密度,并根据数据规模选择合适的网络深度,避免过拟合。
\end{enumerate}

未来的工作可以进一步探究不同激活函数、不同初始化方法以及更复杂的算子(如非线性 PDE 算子)对 DeepONet 性能的影响。

\end{document}
