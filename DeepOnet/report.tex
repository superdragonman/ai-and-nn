\documentclass[a4paper,12pt]{ctexart}
\usepackage{amsmath, amssymb, amsthm}
\usepackage{graphicx}
\usepackage{geometry}
\usepackage[hidelinks]{hyperref}
\usepackage{listings}
\usepackage{xcolor}
\usepackage{float}
\usepackage{indentfirst}
\usepackage{booktabs}
\usepackage{longtable}
\usepackage{cite}

\geometry{left=2.5cm, right=2.5cm, top=2.5cm, bottom=2.5cm}

\title{\textbf{论文阅读报告:\\DeepONet: Learning nonlinear operators for identifying differential equations based on the universal approximation theorem of operators}}
\author{刘昭阳 25215133 \\ 中山大学数学学院(珠海)}
\date{2025年12月22日}

\begin{document}
\maketitle

\begin{abstract}
    本文是对论文《Learning nonlinear operators via DeepONet based on the universal approximation theorem of operators》的深度阅读报告。该论文由Lu Lu, Pengzhan Jin和George Em Karniadakis等人撰写,提出了一种名为DeepONet的深度神经网络架构,旨在学习连续非线性算子。不同于传统的神经网络通常用于逼近有限维空间之间的函数映射,DeepONet基于算子通用近似定理,能够学习从一个无限维函数空间到另一个无限维函数空间的映射。本文将详细阐述DeepONet的理论基础、网络架构、实验结果,并结合个人理解探讨其带来的启发以及未来的衍生想法。
\end{abstract}

\tableofcontents
\newpage

\section{引言}
在科学计算和工程领域,理解和模拟复杂系统的行为至关重要。许多物理系统,如流体力学、电磁学和量子力学系统,都由偏微分方程(PDEs)或积分方程描述。传统的数值方法(如有限元法、有限差分法)虽然精确,但在处理高维问题或需要实时预测的场景下,计算成本往往过高。

近年来,深度学习在函数逼近方面取得了巨大成功。然而,大多数现有的神经网络模型(如CNN, RNN)主要设计用于学习有限维空间之间的映射(例如,图像分类中的像素到标签)。在物理建模中,我们往往需要学习的是一个“算子”(Operator),即从一个函数(如初始条件、边界条件或源项)到另一个函数(如系统解)的映射。

DeepONet的提出旨在应对这一挑战。它不仅仅是针对某一个特定的方程解进行拟合,而是试图学习算子本身。一旦训练完成,DeepONet可以针对任意新的输入函数快速给出系统的响应,而无需重新求解复杂的微分方程。这种“一次训练,多次使用”的特性,使其在代理模型(Surrogate Modeling)和实时控制中具有巨大的潜力。

\section{相关工作}
在DeepONet提出之前,已有许多尝试利用神经网络求解微分方程或学习算子的工作。其中最具代表性的是Raissi等人提出的Physics-Informed Neural Networks (PINNs)。PINNs通过将PDE残差加入损失函数,成功求解了正向和反向问题。然而,PINNs主要针对单一特定的方程实例,一旦边界条件或初始条件改变,网络通常需要重新训练,这限制了其在实时应用中的效率。

除了DeepONet,近年来还出现了Fourier Neural Operator (FNO) 等神经算子方法。FNO在频域内进行卷积操作,具有很好的分辨率无关性。与FNO相比,DeepONet的理论基础更加直接地源于算子通用近似定理,且架构更加灵活,易于处理复杂几何形状。此外,DeepONet也可以被视为一种非线性的降阶模型(Reduced Order Modeling, ROM)。传统的ROM(如POD, DMD)通过线性基函数投影来降低计算复杂度,而DeepONet的Branch Net学习的是非线性特征系数,Trunk Net学习的是非线性基函数,从而能够捕捉更复杂的系统动力学特征。

\section{DeepONet 方法论}

\subsection{理论基础:算子通用近似定理}
DeepONet的核心理论依据是Chen \& Chen在1995年提出的算子通用近似定理(Universal Approximation Theorem for Operators)。该定理指出,对于任意非线性连续算子 $G$,都可以用一个单隐层的神经网络以任意精度逼近。

\textbf{定理 1 (算子通用近似定理)}: 假设 $\sigma$ 是一个连续非多项式函数,$X$ 是一个Banach空间,$K_1 \subset X$ 和 $K_2 \subset \mathbb{R}^d$ 是紧集。$V$ 是 $C(K_1)$ 中的紧集。$G: V \to C(K_2)$ 是一个连续非线性算子。那么对于任意 $\epsilon > 0$,存在正整数 $n, p, m$,以及常数 $c_i^k, \xi_{ij}^k, \theta_i^k, \zeta_k, w_k \in \mathbb{R}$,使得对于任意 $u \in V$ 和 $y \in K_2$,有:
\begin{equation}
    |G(u)(y) - \sum_{k=1}^{p} \sum_{i=1}^{n} c_i^k \sigma \left( \sum_{j=1}^{m} \xi_{ij}^k u(x_j) + \theta_i^k \right) \sigma(w_k \cdot y + \zeta_k)| < \epsilon
\end{equation}
DeepONet的架构正是基于此定理的简化形式,其中Branch Net逼近输入函数 $u$ 的泛函部分,Trunk Net逼近输出域 $y$ 的函数部分。

\subsection{网络架构}
基于上述定理,DeepONet设计了独特的双塔结构,分为“未堆叠”(Unstacked) 和“堆叠”(Stacked) 两种变体。最常用的是未堆叠结构,示意如图 \ref{fig:deeponet-arch} 所示:
\begin{enumerate}
    \item \textbf{Branch Net}: 接收输入函数 $u$ 在 $m$ 个固定传感器位置 $\{x_1, x_2, \dots, x_m\}$ 上的观测值 $[u(x_1), u(x_2), \dots, u(x_m)]$ 作为输入。其输出为一组系数 $[b_1, b_2, \dots, b_p]$。
    \item \textbf{Trunk Net}: 接收输出函数评估点 $y$ 的坐标作为输入。其输出为一组基函数值 $[t_1, t_2, \dots, t_p]$。
\end{enumerate}

最终的预测输出 $G(u)(y)$ 是这两个网络输出的点积(Dot Product):
\begin{equation}
    G(u)(y) \approx \sum_{k=1}^{p} b_k \cdot t_k + b_0
\end{equation}
这种架构巧妙地将输入函数的特征提取与输出域的空间依赖性解耦,极大地提高了模型的表达能力和泛化能力。通常会在最后加上一个偏置项 $b_0$ 以提高拟合能力。

\begin{figure}[H]
    \centering
    \includegraphics[width=0.95\textwidth]{figs/deeponet_overview.png}
    \caption{DeepONet 架构示意图,包含输入/输出、训练数据生成流程,以及未堆叠与堆叠两种实现。}
    \label{fig:deeponet-arch}
\end{figure}

\subsection{数据生成与训练}
为了训练DeepONet,需要生成大量的 (输入函数, 输出函数) 对。
\begin{itemize}
    \item \textbf{输入函数空间采样}: 通常使用高斯随机场(Gaussian Random Fields, GRF)来生成多样化的输入函数 $u(x)$,以覆盖感兴趣的函数空间。GRF的协方差核函数通常定义为径向基函数(RBF):
    \begin{equation}
        k(x_1, x_2) = \exp \left( - \frac{||x_1 - x_2||^2}{2l^2} \right)
    \end{equation}
    其中 $l$ 是相关长度参数,控制生成函数的平滑度。
    \item \textbf{标签生成}: 对于每一个生成的 $u(x)$,使用高精度的传统数值求解器(如FEM, Spectral Methods)求解对应的微分方程,得到真实的解 $G(u)(y)$ 作为标签。
    \item \textbf{损失函数}: 通常使用均方误差(MSE)作为损失函数,最小化预测值与真实值在训练点上的差异。
    $$ \mathcal{L} = \frac{1}{N} \sum_{i=1}^N \sum_{j=1}^P |G(u^{(i)})(y_j) - \hat{G}(u^{(i)})(y_j)|^2 + \lambda ||\Theta||^2 $$
    其中 $\lambda$ 是正则化参数,用于防止过拟合。
\end{itemize}

\section{实验结果与分析}

论文中进行了一系列系统性的实验,涵盖了从简单的常微分方程(ODEs)到复杂的偏微分方程(PDEs)。

\subsection{动态系统与积分算子}
在1D问题中,DeepONet展示了极高的精度。首先是积分算子的学习,即 $G(u)(x) = \int_0^x u(t) dt$。尽管这是一个线性算子,但它验证了DeepONet能够准确捕捉积分的累积效应。其次,论文研究了重力摆 (Gravity Pendulum) 系统,考虑如下非线性ODE:
\begin{equation}
    \frac{d^2s}{dt^2} + \sin(s) = u(t), \quad s(0) = \frac{ds}{dt}(0) = 0
\end{equation}
其中 $u(t)$ 是外力。DeepONet的任务是学习从外力 $u(t)$ 到位移 $s(t)$ 的映射。作者对 $u(t)$ 做密集时间采样(数量级约 $10^2$ 的传感器点),并在大量不同外力轨迹上训练,结果表明模型不仅能拟合训练数据,还能较好泛化到未见过的外力函数。

\subsection{偏微分方程 (PDEs)}
在更复杂的2D PDE问题中,DeepONet同样表现出色。以Darcy Flow为例,模型学习从渗透率场 $K(x)$ 到压力场 $P(x)$ 的映射。方程为:
\begin{equation}
    -\nabla \cdot (K(x) \nabla P(x)) = f(x), \quad x \in (0,1)^2
\end{equation}
这是一个典型的非线性算子学习问题。实验显示,DeepONet能够处理具有剧烈变化的渗透率场,并准确预测压力分布。此外,在对流扩散方程 (Advection-Diffusion) 中,DeepONet也成功学习了不同源项下的浓度分布。方程形式为:
\begin{equation}
    \frac{\partial s}{\partial t} + v \frac{\partial s}{\partial x} = D \frac{\partial^2 s}{\partial x^2}
\end{equation}
结果证明DeepONet能够准确捕捉波的传播和扩散过程。

\subsection{对比实验}
论文将DeepONet与传统的全连接神经网络(FNN)进行了对比。在FNN中,输入通常是将 $u(x)$ 和 $y$ 拼接在一起的向量。实验结果显示,在相同的参数量下,DeepONet的测试误差比FNN低一个到两个数量级。这表明DeepONet的架构引入了适合算子学习的归纳偏置(Inductive Bias),即算子的输出可以分解为输入相关系数与空间基函数的乘积。

\subsection{误差收敛性分析}
论文不仅展示了实验结果,还提供了理论上的误差界。实验观察到,DeepONet的测试误差随着训练样本数量的增加呈现多项式甚至指数级的下降。此外,误差还与传感器数量 $m$ 有关,更多的传感器能提供更丰富的输入函数信息,从而降低近似误差。

\section{个人启发}

阅读这篇论文给我带来了多方面的启发,不仅限于具体的算法细节,更在于解决问题的思维方式。

\subsection{架构即先验 (Architecture as Prior)}
DeepONet的成功再次证明了神经网络架构设计的重要性。通用的MLP虽然理论上也是通用近似器,但在实际学习算子时效率低下。DeepONet通过显式地构造 Branch 和 Trunk 两个分支,实际上是强加了一种物理上的先验知识:即算子的输出可以看作是输入函数特征与空间基函数的线性组合。这种结构与分离变量法、谱方法等传统数学物理方法有着深刻的同构性。这启示我们在设计AI for Science模型时,不应盲目堆砌层数,而应深入思考问题的数学结构,将其融入网络设计中。

\subsection{从“解方程”到“学算子”的范式转变}
传统的科学计算关注于求解单个方程实例。而DeepONet代表了一种范式转变:从求解单一实例转向学习方程背后的算子规律。这种转变类似于从“死记硬背”到“掌握规律”。虽然训练过程可能昂贵(需要大量数据),但一旦学会,推理成本极低。这对于需要反复求解同一类方程的反问题优化、不确定性量化等任务具有革命性意义。

\subsection{数据驱动与物理信息的互补}
虽然DeepONet主要是一个数据驱动的方法,但它并不排斥物理信息。论文中提到的数据生成过程依赖于传统求解器,这本身就是一种物理知识的注入。此外,DeepONet的框架很容易扩展为Physics-Informed DeepONet,即在损失函数中加入PDE残差项,从而减少对标签数据的依赖。这种混合建模的思路是未来科学智能(AI4S)发展的主流方向。

\section{衍生想法}

基于DeepONet的框架,我认为可以在以下几个方向进行深入探索和改进:

\subsection{1. 架构层面的改进}
首先是Attention机制的引入。目前的Branch Net通常使用MLP或CNN,但对于复杂的输入函数,不同区域的重要性可能不同。引入Self-Attention机制(如Transformer Encoder)作为Branch Net,可以更好地捕捉输入函数的长程依赖和局部奇异性。例如,在流体力学中,激波位置附近的输入特征对全局流场影响巨大,Attention机制可以自动聚焦于这些区域。

其次是Trunk Net的基函数优化。Trunk Net负责学习空间基函数,可以尝试使用傅里叶特征映射(Fourier Feature Mapping)或SIREN(Sinusoidal Representation Networks)作为Trunk Net,以增强模型对高频细节和导数信息的拟合能力。这对于求解高波数的波动方程尤为重要。

最后是自适应传感器布局。目前的传感器位置 $x_i$ 通常是固定的。可以设计一种机制,让网络在训练过程中自动学习最优的传感器位置,或者针对不同的输入函数动态调整采样点,从而提高采样效率。

\subsection{2. 训练策略与数据效率}
在训练策略方面,可以引入课程学习 (Curriculum Learning)。算子学习的难度往往与输入函数的复杂度和PDE的非线性程度有关。可以设计课程学习策略,先让网络学习平滑、简单的函数映射,然后逐渐增加输入函数的频率和幅度,引导网络逐步掌握复杂算子。

此外,考虑到数据生成代价高昂,可以使用主动学习 (Active Learning) 策略。通过评估模型在未标注数据上的不确定性,有针对性地生成那些模型“最困惑”的样本进行标注,从而以最少的数据量达到最大的性能提升。

\subsection{3. 应用场景拓展}
DeepONet的应用场景可以进一步拓展到多物理场耦合。现实中的物理问题往往涉及多个物理场的耦合(如流固耦合、热流耦合)。可以设计Multi-Branch DeepONet,分别编码不同的物理场输入,然后在Trunk Net端进行融合,实现多物理场系统的快速模拟。

在反问题求解方面,DeepONet不仅可以用于正向预测,也可以用于反问题。例如,已知系统响应 $G(u)$,求解输入 $u$。可以通过冻结训练好的DeepONet参数,将输入 $u$ 作为可优化变量,通过梯度下降反向求解。或者直接训练一个Inverse DeepONet,交换Branch和Trunk的角色。

最后是几何泛化。原始DeepONet主要处理规则区域。对于复杂几何形状,可以将几何信息(如SDF, Signed Distance Function)也作为Branch Net的输入,或者使用Graph Neural Networks (GNN) 作为Trunk Net来处理非结构化网格,从而实现对不同几何形状的泛化能力。

\section{局限与注意事项}
尽管DeepONet表现出强大的算子学习能力,实践中仍需注意以下局限:
\begin{itemize}
    \item \textbf{传感器依赖}: Branch Net依赖固定的采样点 $x_i$,当测试时的输入函数具有新的局部奇异或分辨率需求时,性能可能下降。
    \item \textbf{分布外泛化}: 训练集通常来自有限的随机场或参数分布,超出分布的输入函数(更高频或更剧烈的非线性)会显著降低精度。
    \item \textbf{边界/物理约束}: 基础DeepONet未显式编码PDE约束,若标签数据稀缺,建议结合Physics-Informed损失以提升稳定性。
    \item \textbf{高维输出成本}: 对高维 $y$ 域或长时间演化问题,Trunk Net所需的基函数数目 $p$ 可能迅速增大,导致内存与训练时间开销增加。
\end{itemize}

\section{结论}
DeepONet作为一种基于算子通用近似定理的深度学习框架,为非线性算子的学习提供了一种高效、通用的解决方案。它不仅在理论上具有坚实的基础,在实验中也展现出了优越的性能和泛化能力。通过将输入函数编码与输出空间表示解耦,DeepONet成功地打破了传统神经网络在函数空间映射上的局限。

本文详细回顾了DeepONet的原理与贡献,并结合个人思考提出了若干改进方向。随着AI与科学计算的深度融合,相信DeepONet及其衍生变体将在流体力学、材料科学、气候预测等领域发挥越来越重要的作用。它不仅是一个高效的计算工具,更是连接数据科学与物理定律的一座桥梁。

\newpage
\end{document}
